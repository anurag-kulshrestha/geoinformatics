%%%%%%%%%%%%%%%%%
% This is an sample CV template created using altacv.cls
% (v1.1.2, 1 February 2017) written by LianTze Lim (liantze@gmail.com). Now compiles with pdfLaTeX, XeLaTeX and LuaLaTeX.
%
%% It may be distributed and/or modified under the
%% conditions of the LaTeX Project Public License, either version 1.3
%% of this license or (at your option) any later version.
%% The latest version of this license is in
%%    http://www.latex-project.org/lppl.txt
%% and version 1.3 or later is part of all distributions of LaTeX
%% version 2003/12/01 or later.
%%%%%%%%%%%%%%%%

%% If you need to pass whatever options to xcolor
\PassOptionsToPackage{dvipsnames}{xcolor}

%% If you are using \orcid or academicons
%% icons, make sure you have the academicons
%% option here, and compile with XeLaTeX
%% or LuaLaTeX.
% \documentclass[10pt,a4paper,academicons]{altacv}

%% Use the "normalphoto" option if you want a normal photo instead of cropped to a circle
% \documentclass[10pt,a4paper,normalphoto]{altacv}

\documentclass[10pt,a4paper]{altacv}

%% AltaCV uses the fontawesome and academicon fonts
%% and packages.
%% See texdoc.net/pkg/fontawecome and http://texdoc.net/pkg/academicons for full list of symbols.
%%
%% Compile with LuaLaTeX for best results. If you
%% want to use XeLaTeX, you may need to install
%% Academicons.ttf in your operating system's font
%% folder.


% Change the page layout if you need to
\geometry{left=1cm,right=9cm,marginparwidth=6.8cm,marginparsep=1.2cm,top=1.25cm,bottom=1.25cm}

% Change the font if you want to.

% If using pdflatex:
\usepackage[utf8]{inputenc}
\usepackage[T1]{fontenc}
\usepackage[default]{lato}

% If using xelatex or lualatex:
% \setmainfont{Lato}

% Change the colours if you want to
\definecolor{Mulberry}{HTML}{72243D}
\definecolor{SlateGrey}{HTML}{2E2E2E}
\definecolor{LightGrey}{HTML}{666666}
\colorlet{heading}{Sepia}
\colorlet{accent}{Mulberry}
\colorlet{emphasis}{SlateGrey}
\colorlet{body}{LightGrey}

% Change the bullets for itemize and rating marker
% for \cvskill if you want to
\renewcommand{\itemmarker}{{\small\textbullet}}
\renewcommand{\ratingmarker}{\faCircle}

%% sample.bib contains your publications
\addbibresource{sample_1.bib}

\begin{document}
\name{Anurag Kulshrestha}
\tagline{M.Sc. - Geoinformatics Student at ITC, University of Twente}
\photo{2.8cm}{anurag}
\personalinfo{%
  % Not all of these are required!
  % You can add your own with \printinfo{symbol}{detail}
  \email{anurag.librian@gmail.com}
  \phone{+917617611181}
  \mailaddress{80, Pushpanjali Vatika, Sikandra}
  \location{Agra, INDIA}
  \homepage{anuragiirs.wordpress.com}
  \twitter{@anurag41091}
  \linkedin{linkedin.com/in/anurag-kulshrestha-38739478}
  \github{github.com/hzsmhzsm}
  %% You MUST add the academicons option to \documentclass, then compile with LuaLaTeX or XeLaTeX, if you want to use \orcid or other academicons commands.
%   \orcid{orcid.org/0000-0000-0000-0000}
}

%% Make the header extend all the way to the right, if you want.
\begin{fullwidth}
\makecvheader
\end{fullwidth}

%% Provide the file name containing the sidebar contents as an optional parameter to \cvsection.
%% You can always just use \marginpar{...} if you do
%% not need to align the top of the contents to any
%% \cvsection title in the "main" bar.
\cvsection[sample-p1sidebar]{Current Study}
\cvevent{M.Sc.\ in Geographic Information Science and Earth Observation}{ITC, University of Twente}{Sept 2016 -- March 2018}{Enschede, Netherlands}
\begin{itemize}
\item JEP with Indian Institute of Remote Sensing (IIRS)-ISRO, Dehradun
\item Specialization: Geoinformatics 
\item Thesis Topic: Detection and Characterization of Oil Spills using Polarimetric SAR Remote Sensing
\end{itemize}

\cvsection{Experience}


\cvevent{Logistics Management Software Developer}{Kumar Roadlines}{Feb 2016 -- April 2016}{Agra, India}
\begin{itemize}
\item Developed a logistics management system based on Java and Sqlite database
\item Fully functional and presently helping in the management of a small fleet of tankers in Northern India.
\end{itemize}

\divider

\cvevent{Attempt at a startup}{CAC Systems}{Jan 2015 -- Jan 2016}{Agra, India}
\begin{itemize}
\item Theme: Biorefienry based remediation of wastewater
\item Incubator and Investor: Ashoka Chemicals, Agra
\end{itemize}


\cvsection{Projects}
\cvevent{M.Sc. Dissertation: Detection and Characterization of Oil Spills using PolSAR Remote Sensing}{ITC, IIRS}{Sept 2017 - March 2018}{Dehradun, India}
\begin{itemize}
\item Multi-frequency analysis and developent of oil-spill detection model using UAVSAR, RISAT-1 and TerraSAR-X data.
\item Model built using geospatial python libraries e.g. \textit{gdal}, \textit{ogr} and \textit{scipy} and  open-source softwares, e.g. \textit{PolSARPro}, \textit{QGIS}
\item An oil spill detection tool using Qt-Designer is also under development.
\end{itemize}

\divider

\cvevent{Bioremediation of Wastewater using Microalgae}{Indian Institute of Science (IISc)}{August 2015 -- December 2015}{Bangalore, India}
\begin{itemize}
\item Case study on the feasibiity of decentralized wastewater treatment using organic techniques such as aquaculture and hydroponics.
\end{itemize}

\divider

\cvevent{M.Sc. Dissertation: Neutral Atom Traps and Bose-Hubbard Model}{Indian Institute of Science Education and Research}{August 2013 -- April 2014}{Bangalore, India}
\begin{itemize}
\item Domain: Cold-atom physics
\item Modelling of magnetic neutral atom traps and 2 dimentional lattice to trap bosons in Mathematica.
\end{itemize}


%\medskip

\cvsection{A Day of My Life}

% Adapted from @Jake's answer from http://tex.stackexchange.com/a/82729/226
% \wheelchart{outer radius}{inner radius}{
% comma-separated list of value/text width/color/detail}
\wheelchart{1.5cm}{0.5cm}{%
  7/8em/accent!30/{Sleep},
  2/8em/accent!40/{Sports, Yoga and Meditation},
  8/8em/accent!60/Daytime job,
  2/8em/accent!60/Time with Friends,
  2/10em/accent/Recreational Learning,
  3/6em/accent!20/Night-time work
}

\defbibfilter{papers}{
  type=article
}


%\clearpage
\cvsection[page2sidebar]{Publications} 	
%\cvsection{Publications}
\nocite{*}
%\printbibliography
\printbibliography[heading=pubtype,title={\printinfo{\faBook}{Books}},type=book]

%\divider

\printbibliography[heading=pubtype,title={\printinfo{\faFileTextO}{Journal Articles}},type=article]
%\printbibliography[heading=pubtype,title={\printinfo{\faFileTextO}{Journal Articles}},filter=papers]
%\divider

\printbibliography[heading=pubtype,title={\printinfo{\faGroup}{Conference Proceedings}},type=inproceedings]


\faGroup 
\textbf{Conference Proceedings}
\vspace{.3cm}
\begin{itemize}
\item Kulshrestha, A. (2017). Dark Spot Detection and Characterization of Marine Surface Slicks using PolSAR Remote Sensing. Paper presented at Asian Conference on Remote Sensing-2017, New Delhi, India.
\end{itemize}
%% If the NEXT page doesn't start with a \cvsection but you'd
%% still like to add a sidebar, then use this command on THIS
%% page to add it. The optional argument lets you pull up the
%% sidebar a bit so that it looks aligned with the top of the
%% main column.
%\addnextpagesidebar[-1ex]{page2sidebar}

\end{document}
